\documentclass[a4paper,11pt]{article}

\htmlcss{style.css}
\htmltitle{Camllib}
\htmlpanel{0}
\setcounter{htmlautomenu}{1}
\setcounter{htmldepth}{1}

\usepackage{hyperlatex}
\usepackage{xspace}
%\usepackage{frames}

\newcommand{\ocaml}{\xlink{OCaml}{http://www.caml.org}\xspace}

\title{Camllib}
\date{}
\author{}

\begin{document}

%\xmlattributes*{img}{align="left"}
%\xlink{\htmlimg{http://devel.inria.fr/logo_inria.png}{INRIA}}{http://www.inria.fr}
\xlink{Up}{../index.html}
\maketitle

\section{About}

Camllib is an \ocaml library implementing various abstract
datatypes inspired by the standard OCaml library. It provides:
\begin{itemize}
\item Slightly improved versions of Set, Map, Hashtbl modules,
  named Sette, Mappe, Hashhe. These versions offers the choice
  between polymorphic types, or specialization via a functor. A
  few additional functions are offered (functions Mappe.maptoset,
  ...).
\item Two-ways associations, Multisets, Union-Find, Graphs.
\item Various small modules 
\item Standard specialization are also provided, such as HashheI
  for hashtables over integers.
\end{itemize}

\section{License}

Files that are modifications of files of the \ocaml distribution
inherit the LGPL license of their original version in the \ocaml
system. Other are also released under the LGPL license.

\section{Download}

\begin{itemize}
\item \xlink{Subversion repository}{http://gforge.inria.fr/plugins/scmsvn/viewcvs.php/?root=bjeannet}, \xlink{current version: 1.2.0}{http://gforge.inria.fr/plugins/scmsvn/viewcvs.php/pkg/fixpoint/branches/release-1.2.0/?root=bjeannet} (see \xlink{Up}{../index.html} for the organisation of the repository)
\end{itemize}

\section{Documentation}
\begin{itemize}
\item \xlink{On-line}{html/index.html}
\item \xlink{PDF}{camllib.pdf}
\end{itemize}

\end{document}
