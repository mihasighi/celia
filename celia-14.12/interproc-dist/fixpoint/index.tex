\documentclass[a4paper,11pt]{article}

\usepackage{hyperlatex}
\usepackage{xspace}
%\usepackage{frames}
\W\htmlcss{style.css}
\W\htmltitle{Fixpoint}
\W\htmlpanel{0}
\setcounter{htmlautomenu}{1}
\setcounter{htmldepth}{1}

\newcommand{\ocaml}{\xlink{OCaml}{http://www.caml.org}\xspace}
\newcommand{\camlmod}[1]{\xlink{#1}{html/#1.html}\xspace}

\title{Fixpoint}
\date{}
\author{}

\begin{document}

%\xmlattributes*{img}{align="left"}
%\xlink{\htmlimg{http://devel.inria.fr/logo_inria.png}{INRIA}}{http://www.inria.fr}

\xlink{Up}{../index.html}
\par

\maketitle

\section{About}

Fixpoint is an \ocaml library implementing a generic fixpoint
engine. The interface is parameterized by the abstract domain on
which fixpoint computations are performed.

A short presentation in \xlink{french}{presentation-fixpoint.pdf}

The fixpoint solver library is composed of several modules.
\begin{itemize}
\item \camlmod{Fixpoint}: this is the only module to look at for a
  normal user.
\item \xlink{example.ml}{example.ml}: example of use of the library;
\item \camlmod{FixpointType}: defines the various types and
  associated printing functions (including for the DOT output);
\item \camlmod{FixpointStd}: implements the Kleene-Bourdoncle
  iteration technique \cite{bourdoncle93} (combined with
  working-set algorithm) for solving equations described
  explicitly by an hypergraph, and offers base functions for the
  more sophisticated techniques;
\item \camlmod{FixpointGuided}: implements the Gopan-Reps guided
  iteration technique \cite{GopanReps07} for equations described
  explicitly by an hypergraph;
\item \camlmod{FixpointDyn}: exploits an implicit description
  (under the form if a successor function) of the equation graph,
  which is then explored dynamically by alternating propagation
  phase (to detect newly non-empty variables) and
  upto-convergence-iteration phase.
\end{itemize}

\section{License}

LGPL license (GNU Library General Public License).

\section{Download}

\begin{itemize}
\item \xlink{Subversion repository}{http://gforge.inria.fr/plugins/scmsvn/viewcvs.php/?root=bjeannet}, (current version: 3.0) (see \xlink{Up}{../index.html} for the organisation of the repository)
\end{itemize}

\section{Documentation}
\begin{itemize}
\item \xlink{PDF}{fixpoint.pdf} with a small introduction
\item \xlink{On-line}{html/index.html}
\item \xlink{Example file}{example.ml}
\end{itemize}

\bibliography{bib.bib,mybib.bib}
\bibliographystyle{alpha}
\end{document}
