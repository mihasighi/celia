\documentclass[twoside,10pt,a4paper]{report}
\usepackage[latin1]{inputenc}
\usepackage[T1]{fontenc}
\usepackage{ae}
\usepackage{fullpage}
\usepackage{pstricks,pst-node}
\newrgbcolor{lightblue}{0.92 0.96 1}
\newrgbcolor{lightgreen}{0.92 1 0.86}
\newrgbcolor{lightred}{1 0.9 0.8}
\newrgbcolor{darkgreen}{0 0.3 0}
\newcommand{\tuple}[1]{\langle #1 \rangle}

\usepackage{url}
\usepackage{ocamldoc}
\usepackage{makeidx}

\usepackage{fancyhdr}
\pagestyle{fancy}
\renewcommand{\headrulewidth}{0.9pt}
\renewcommand{\footrulewidth}{0pt}
\setlength{\headheight}{2.8ex}
\setlength{\footskip}{5ex}
\renewcommand{\chaptermark}[1]{ %
  \markboth{\MakeUppercase{\chaptername}\ \thechapter.\ #1}{}}
\renewcommand{\sectionmark}[1]{}
\setcounter{tocdepth}{2}
\setcounter{secnumdepth}{2}
\usepackage{color}
\definecolor{mygreen}{rgb}{0,0.6,0}

\usepackage[ps2pdf]{hyperref}

\setlength{\parindent}{0em}
\setlength{\parskip}{0.5ex}

\makeindex

\title{Fixpoint 3.0}
\author{Bertrand Jeannet}
\begin{document}
\maketitle

\vspace*{0.9\textheight}

The FIXPOINT library is distributed under LGPL license.

Copyright (C) Bertrand Jeannet 2004-2009

\newpage

\tableofcontents

% **********************************************************************
\chapter{Overview}
% **********************************************************************

% ======================================================================
\section{Organization}
% ======================================================================

The fixpoint solver library is composed of the following modules:
\begin{itemize}\setlength{\itemsep}{0pt}
\item \texttt{Fixpoint}: this is the only module to look at for a
  normal user.
\item \texttt{Example}: example of use of the library;
\medskip

\item \texttt{FixpointType}: defines the various types and
  associated printing functions (including for the DOT output);
\item \texttt{FixpointStd}: implements the Kleene-Bourdoncle
  iteration technique \cite{bourdoncle93} (combined with
  working-set algorithm) for solving equations described
  explicitly by an hypergraph, and offers base functions for the
  more sophisticated techniques;
\item \texttt{FixpointGuided}: implements the Gopan-Reps guided
  iteration technique \cite{GopanReps07} for equations described
  explicitly by an hypergraph;
\item \texttt{FixpointDyn}: exploits an implicit description
  (under the form if a successor function) of the equation graph,
  which is then explored dynamically by alternating propagation
  phase (to detect newly non-empty variables) and
  upto-convergence-iteration phase.
\end{itemize}

We also include in this documentation the modules from the
\texttt{camllib} used by \texttt{fixpoint}.
\begin{itemize}\setlength{\itemsep}{0pt}
\item \texttt{SHGraph}: hypergraphs (used to describe equation systems);
\item \texttt{Ilist}: imbricated lists (used for strongly connected sub-components);
\item \texttt{Sette}: generalizes the \texttt{Set} module of the
  OCaml standard library;
\item \texttt{Hashhe}: generalizes the \texttt{Hashtbl} module of the
  OCaml standard library;
\item \texttt{Print}: provides printing functions for list,
  arrays, \ldots using the \texttt{Format} module;
\item \texttt{Time}: for measuring times
\end{itemize}

% ======================================================================
\section{Overview}
% ======================================================================

\subsection{Equation system described explicitly by an hypergraph}

\newcommand{\vtx}[1]{\pscirclebox[linecolor=green,fillstyle=solid,fillcolor=lightgreen]{#1}}
\newcommand{\edg}[1]{\psframebox[linecolor=blue,fillstyle=solid,fillcolor=lightblue]{#1}}

\begin{figure}[t]

  \parbox{0.47\textwidth}{
  \vspace*{2.5ex}%
  \begin{center}\
    \psset{mnode=oval,emnode=p,linewidth=0.01pt}
    \psset{linestyle=solid,fillcolor=lightgreen,linecolor=green,fillstyle=solid}
    \begin{psmatrix}[rowsep=5ex,colsep=8em]
      ~$c_0$~ \\
      & ~$c_2$~ \\[3.3ex]
      ~$c_1$~ & ~$c_3$~ \\[3.3ex]
      & ~$c_4$~ \\
      ~$c_5$~
    \end{psmatrix}
    \psset{arrows=->,linestyle=solid,arrowsize=2pt 3,arrowinset=0.05}
    \psset{linewidth=0.8pt,fillstyle=none,linecolor=black}
    \ncline{1,1}{3,1}\nbput{$\tuple{(n = 0)?}$}
    \ncline{3,1}{5,1}\nbput{$\tuple{r := 1}$}
    \ncarc{1,1}{2,2}\aput{:U}{$\tuple{(n > 0)?}$}
    \ncline{2,2}{3,2}\naput{$\tuple{x := n \!-\! 1}$}
    \psset{arrows=<-}
    \ncarc[ncurv=1]{4,2}{5,1}\aput{:D}{$\tuple{\mathtt{ret}\, r := \mathtt{fact}(x)}$}
    \nccurve[angleA=-45,angleB=180]{1,1}{3,2}
    \aput{:U}{$\tuple{\mathtt{call}\, r := \mathtt{fact}(x)}$}
    \ncarc[ncurv=1]{5,1}{4,2}\aput{:U}{$\tuple{r := r \!*\! n}$}
    \psset{arrows=->}
    \psset{linewidth=0.4pt,linestyle=dashed,linecolor=blue}
    \ncline{3,2}{4,2}\naput{\blue$\tuple{r := \mathtt{fact}(x)}$}
  \end{center}
  \begin{center}
    \caption{Control-Flow-Graph of the Factorial program}
    \label{fig:cfg}
  \end{center}
  }
  \hfill
  \parbox{0.47\textwidth}{
  \begin{center}
    \hspace*{-2em}
    \psset{linestyle=solid,fillstyle=solid}
    \begin{psmatrix}[mnode=r,rowsep=2ex,colsep=3em]
      & \vtx{$0$} \\
      && \edg{$2$} \\
      \edg{$1$} & \edg{$5$} & \vtx{$2$} \\
      &&  \edg{$3$} \\
      \vtx{$1$} && \vtx{$3$} \\
      && \edg{$4$} \\
      \edg{$6$} && \vtx{$4$} \\
      && \edg{$7$} \\
      & \vtx{$5$}
    \end{psmatrix}
    \nput{180}{3,1}{$\tuple{(n = 0)?}$}
    \nput{180}{7,1}{$\tuple{r := 1}$}
    \nput{0}{2,3}{$\tuple{(n > 0)?}$}
    \nput{0}{4,3}{$\tuple{x := n \!-\! 1}$}
    \nput{0}{6,3}{$\tuple{\mathtt{ret}\, r := \mathtt{fact}(x)}$}
    \nput*{-90}{3,2}{$\tuple{\mathtt{call}\, r := \mathtt{fact}(x)}$}
    \nput{0}{8,3}{$\tuple{r := r \!*\! n}$}
   %
    \psset{arrows=->,linestyle=solid,arrowsize=2pt 3,arrowinset=0.05}
    \psset{linewidth=0.8pt,fillstyle=none,linecolor=black}
    \ncline[arrows=-]{1,2}{3,1}
    \ncline{3,1}{5,1}
    \ncline[arrows=-]{5,1}{7,1}
    \ncline{7,1}{9,2}
    %
    \ncline[arrows=-]{1,2}{2,3}
    \ncline{2,3}{3,3}
    \ncline[arrows=-]{3,3}{4,3}
    \ncline{4,3}{5,3}
    \nccurve[angleA=150,angleB=-90,arrows=-]{5,3}{3,2}
    \ncline{3,2}{1,2}
    \ncline[arrows=-]{5,3}{6,3}
    \nccurve[angleA=80,angleB=135,arrows=-]{9,2}{6,3}
    \ncline{6,3}{7,3}
    \ncline[arrows=-]{7,3}{8,3}
    \ncline{8,3}{9,2}
  \end{center}
  \begin{center}
    \caption{Hypergraph representing the equation system of Eqn~(\ref{eqn:eqs})}
    \label{fig:hypergraph}
  \end{center}

  }
\end{figure}

A equation system of the form 
\begin{equation}
  \label{eqn:eqs}
  \left\{
  \begin{array}{rcl}
  X_0 &=& X_0^i \sqcup F_0(X_3) \\
  X_1 &=& F_1(X_0) \\
  X_2 &=& F_2(X_1) \\
  X_3 &=& F_3(X_2) \\
  X_4 &=& F_4(X_3,X_5) \\
  X_5 &=& F_5(X_1) \sqcup F_6(X_4)
  \end{array}
  \right.
\end{equation}
as generated for instance by the relational interprocedural
analysis of the factorial program depicted on Fig.~\ref{fig:cfg}
will be represented by an hypergraph, depicted on
Fig.~\ref{fig:hypergraph}, the vertices of which denotes
variables, and the hyperedges functions. Hyperedges are needed for
functions taking multiple arguments, such as the ``combine''
function $F_4$ which models procedure return by combining the
value at the call-site ($X_3$) and the value at the return-site
($X_5$).

The library use the module \texttt{SHGraph} and the type
\texttt{('vertex, 'hedge, 'a,'b,'c) SHGraph.t} to represent such
hypergraphs. This provided datatype distinguishes vertex and
hyperedge identifiers from the information associated to them.

\subsection{Equation system described implicitly by a successor function}

Alternatively, the equation system~(\ref{eqn:eqs}) can be
described by the following function:
\begin{equation}
  \psset{framesep=0.15ex}
  \begin{array}{rcl}
    \vtx{0} &\longmapsto& 
    \begin{array}[t]{rrcll}
      \Bigl\{ & \edg{1} &\mapsto& \left(\left[\vtx{0}\right],~~\vtx{1}\right) \\
         & \edg{2} &\mapsto& \left(\left[\vtx{0}\right],~~\vtx{2}\right) & \Bigr\}
    \end{array} \\
    \vtx{1} &\longmapsto& 
    \begin{array}{rrcll}
      \Bigl\{ & \edg{5} &\mapsto& \left(\left[\vtx{1}\right],~~\vtx{5}\right) & \Bigr\}
    \end{array} \\
    \vtx{2} &\longmapsto& 
    \begin{array}{rrcll}
      \Bigl\{ & \edg{3} &\mapsto& \left(\left[\vtx{2}\right],~~\vtx{3}\right) & \Bigr\}
    \end{array} \\
    \vtx{3} &\longmapsto& 
    \begin{array}{rrcll}
      \Bigl\{ & \edg{4} &\mapsto& \left(\left[\vtx{3};\vtx{5}\right],~~\vtx{4}\right) & \Bigr\}
    \end{array} \\
    \vtx{4} &\longmapsto& 
    \begin{array}{rrcll}
      \Bigl\{ & \edg{6} &\mapsto& \left(\left[\vtx{4}\right],~~\vtx{5}\right) & \Bigr\}
    \end{array} \\
    \vtx{5} &\longmapsto& 
    \begin{array}{rrcll}
      \Bigl\{ & \edg{4} &\mapsto& \left(\left[\vtx{3};\vtx{5}\right],~~\vtx{4}\right) & \Bigr\}
    \end{array}
  \end{array}
\end{equation}
This function associates to each vertex $v$ a map associating to
each successor hyperedge $h$ a pair composed of (i) the array of
predecessor vertex of $h$ (including $v$), and (ii) the successor
vertex of $h$.

The library use the type \texttt{('vertex,'hedge) Fixpoint.equation} to describe
such a function.

\subsection{Interpreting an equation system}

\texttt{Fixpoint} is parameterized by an object of type [('vertex,
'hedge, 'abstract, 'arc) Fixpoint.manager], which defines the type
of vertex and hyperedge identifiers, the type of abstract values
attached to vertices (which is the value of the variable), and the
type of values attached to hyperedges (optional, maybe
\texttt{unit}). The manager defines the needed functions on
abstract values, the function that computes the effect of an
hyperedge, and a set of options and debugging functions.

\subsection{Iteration strategies}

Fixpoint solving iterations are parametrized by an object of type
\texttt{('vertex,'hedge) Fixpoint.strategy}. We refer to
\cite{bourdoncle93} for complete explanations, and to the
documentation of module \texttt{Fixpoint}. In short, a strategy
is an imbricated list like $[1; [[2;3]; 4]; [5]; 6]$, which
means, during the iterative solving:
\begin{enumerate}\setlength{\itemsep}{0pt}
\item update vertex 1;
\item update 2 and 3, and loop until stabilization;
\item update 4, and come back to 2. until stabilization;
\item update 5 and loop until stabilization;
\item update 6 and ends the analysis.
\end{enumerate}
\cite{bourdoncle93} represents such an iteration strategy by the
regular expression $(1~ ((2~3)^*~ 4)^*~ (5^*)~ 6)$. Alternatively, one
could use a flatter iteration strategy like $(1~ (2~ 3~ 4)^*~ (5^*)~ 6)$.

Standard users (as myself) is advised to use the function
\texttt{Fixpoint.make\_strategy} to generate correct strategies
from an hypergraph.

% ======================================================================
\section{Example}
% ======================================================================

I suggest to look at the example file \texttt{example.ml}
and to play a bit with it.

\begin{thebibliography}{Bou93}

\bibitem[Bou93]{bourdoncle93}
F.~Bourdoncle.
\newblock Efficient chaotic iteration strategies with widenings.
\newblock In {\em International Conference on Formal Methods in Programming and
  their Applications}, volume 735 of {\em LNCS}, 1993.

\bibitem[GR07]{GopanReps07}
Denis Gopan and Thomas~W. Reps.
\newblock Guided static analysis.
\newblock In {\em Static Analysis Symposium, SAS'07}, volume 4634 of {\em
  LNCS}, August 2007.

\end{thebibliography}

%\bibliographystyle{alpha}
%\bibliography{bib}

% **********************************************************************
% **********************************************************************
\part{Main fixpoint module}
\input{Fixpoint.tex}

\part{Other fixpoint modules}
\input{FixpointType.tex}
\input{FixpointStd.tex}
\input{FixpointGuided.tex}
\input{FixpointDyn.tex}

\part{Auxiliary modules}

\input{SHGraph}
\input{Ilist}
\input{Sette}
\input{Hashhe}
\input{Print}
\input{Time}

\appendix
\printindex


\end{document}
